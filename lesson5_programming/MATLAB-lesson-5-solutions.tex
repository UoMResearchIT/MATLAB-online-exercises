\documentclass{article}

\usepackage{fullpage} % Less line-wrapping of code
%\usepackage{lmodern}	% Use lmodern font to get the correct display of tilde character for ~=
%\usepackage[T1]{fontenc}
\usepackage{enumitem}
\usepackage{hyperref}
\hypersetup{
  colorlinks   = true, %Colours links instead of ugly boxes
  urlcolor     = blue, %Colour for external hyperlinks
  linkcolor    = black, %Colour of internal links
}

\title{MATLAB lesson 5: Programming\\Solutions to exercises}
\date{}
\author{Dr. Gerard Capes\thanks{Questions and feedback can be directed to \href{mailto:gerard.capes@manchester.ac.uk?subject=Feedback on MATLAB lesson 5 (programming) solutions sheet}{gerard.capes@manchester.ac.uk}}}
\usepackage{matlab-prettifier}

\begin{document}
	\maketitle
	
	\section{Logical operators}
	\subsection*{Based on the lesson}
		\begin{enumerate}
			\setcounter{enumi}{2}
			\item \lstinputlisting[style=Matlab-editor]{q1_3.m}
			\item 
			\begin{itemize}
				\item Create a new script using the \texttt{new script} button on the \texttt{home} tab, or using the keyboard shortcut ctrl+N
				\item Type commands into the script window, then save the script, chosing a descriptive name (something which reflects what the script does)
				\item Make sure the file extension is \texttt{.m} 
				\item You can then run the script by typing the name of the script (without the \texttt{.m} extension), clicking the \texttt{run} button on the \texttt{editor} tab, or pressing \texttt{F5}.
			\end{itemize}
		\end{enumerate}
	\subsection*{Using MATLAB's help}
	\begin{enumerate}[resume]
		\item \lstinputlisting[style=Matlab-editor]{q1_5.m}
		
		\item \lstinputlisting[style=Matlab-editor]{q1_6.m}
		
		\item \lstinputlisting[style=Matlab-editor]{q1_7.m}
	\end{enumerate}
	
	\section{Flow control}
	\subsection*{Based on the lesson}
	\begin{enumerate}
		\item \lstinputlisting[style=Matlab-editor]{q2_1.m}
		
		\item \lstinputlisting[style=Matlab-editor]{q2_2.m}
		
		\item \lstinputlisting[style=Matlab-editor]{q2_3.m}
		
		\item In a \texttt{switch}, each \texttt{case} is looking for a match (i.e. testing the switch variable for a match to the value in each \texttt{case}). When testing an expression, you should use an \texttt{if} instead. A \texttt{switch} is best reserved for cases when you expect the result to be one of a discrete (non-continuous) series of values.
		
		\item \lstinputlisting[style=Matlab-editor]{q2_5.m}
		
		\item \lstinputlisting[style=Matlab-editor]{q2_6.m}
		
		\item Generally speaking, you should use a \texttt{for} loop when you know in advance how many iterations you will require. If you don't know how many iterations you will need, you can use a \texttt{while} loop to iterate for as long as a condition remains true.
		
		\item \lstinputlisting[style=Matlab-editor]{q2_8.m}	
		
		\item \lstinputlisting[style=Matlab-editor]{q2_9.m}	
		
		\item Using a for loop
		\lstinputlisting[style=Matlab-editor]{q2_10_for.m}
		
		Using a while loop
		\lstinputlisting[style=Matlab-editor]{q2_10_while.m}
	\end{enumerate}
	
\end{document}
