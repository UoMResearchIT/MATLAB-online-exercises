\documentclass{article}

\usepackage{lmodern}	% Use lmodern font to get the correct display of tilde character for ~=
\usepackage[T1]{fontenc}
%\usepackage{enumerate}
\usepackage{enumitem}
\usepackage{hyperref}
\hypersetup{
  colorlinks   = true, %Colours links instead of ugly boxes
  urlcolor     = blue, %Colour for external hyperlinks
  linkcolor    = black, %Colour of internal links
}

\title{MATLAB lesson 5: Programming\\Exercise sheet}
\date{}
\author{Dr. Gerard Capes\thanks{Questions and feedback can be directed to \href{mailto:gerard.capes@manchester.ac.uk?subject=Feedback on MATLAB lesson 5 (programming) exercise sheet}{gerard.capes@manchester.ac.uk}}}

\begin{document}
	\maketitle
	
	\section{Logical operators}
	\subsection*{Based on the lesson}
	\begin{enumerate}
		\item Run the tests \texttt{a<b}, \texttt{a>=b}, \texttt{a==b}, and \texttt{a\textasciitilde=b} for the following values of \texttt{a} and \texttt{b}:
		
		\begin{enumerate}
			\item a=1, b=1
			\item a=1, b=2
			\item a=1, b=inf
		\end{enumerate}
		Make sure you understand the result.
		\item For \texttt{a=1}, \texttt{b=1}, \texttt{c=0}, test the following and make sure you understand the results:
		\begin{enumerate}
			\item \texttt{(b | c) \& (a < 2)}
			\item \texttt{(a < b) | (b > c)}
			\item \texttt{(a \& b) \& c}
		\end{enumerate}	
		\item Generate a 500 by 1 array (i.e. 500 rows, 1 column) of random values and test if any values are greater than 0.5 (hint: \texttt{rand}, \texttt{any}).
	\end{enumerate}	
	\subsection*{Also requires some use of MATLAB's help}	
		\begin{enumerate}[resume]
		\item Use the MATLAB help to find out how to write a run a script (hint: \texttt{doc Programming Scripts and Functions}). 
		\item Write a script that generates a 500 by 1 array (i.e. 500 rows, 1 column) of random values (use \texttt{rand}) and does the following:
		\label{q:findarrayindices}
		\begin{enumerate}
			\item Tests if any values are greater than 0.5, 0.9, 0.99
			\item Finds the indices where values are greater than 0.99
			\item Finds if all values are greater than 0.5, 0.1 and 0.01
			\item Sets the values which are greater than 0.99 equal to 1
		\end{enumerate}
		\item Repeat question \ref{q:findarrayindices} for a 10 by 10 matrix of random values
		For each test use the \texttt{disp} function to output sensible text.
		\item Write a script which creates two 5*5 matrices, \texttt{r1} and \texttt{r2}, each filled with random numbers.
		\begin{enumerate}
			\item Test which of the values in \texttt{r}1 are greater than their counterparts in \texttt{r2}.
			\item Test which the values in \texttt{r1} are greater than 0.5, 0.9, 0.99. 
			Your tests should return a 5*5 matrix of logical values.
		\end{enumerate}
	\end{enumerate}
	
	\section{Flow control}
	\label{sec:flowcontrol}
	\subsection*{Based on the lesson}
	\begin{enumerate}
		\item Create a variable \texttt{x} and assign it a value.
		\label{q:ifelse}
		\begin{enumerate}
			\item Write a script that uses an \texttt{if} statement to test whether the variable \texttt{x} has a value in the range $1<x<2$. The script should output a suitable message if it is (use \texttt{disp}).
			\item Modify your script using an \texttt{elseif} to also test if \texttt{x} is less than or equal to 1. Your script should output a suitable message if so.
			\item Add a further test which outputs suitable text when neither of the above conditions are met.
			\item Change the value of \texttt{x} and re-run your script to test whether it works correctly for each condition.
		\end{enumerate}
		\item 
		\label{q:ifclass}
		\begin{enumerate}
			\item Write a script that uses an if statement to test whether the class of a variable \texttt{x} is a double (use the \texttt{isa} function). The script should output a suitable message if it is (use \texttt{disp}).
			\item Modify your script using an \texttt{elseif} to also test for \texttt{char} and \texttt{logical} classes. Your script should output suitable message if \texttt{x} is any of these classes.
			\item Modify your script to output the message ``Unknown class'' when the class of \texttt{A} is none of the above (e.g. \texttt{single}, \texttt{int8}).
			\item Test your script by changing the class of \texttt{x} and re-running the script.
		\end{enumerate}
		\item Write a new script that answers question \ref{sec:flowcontrol}.\ref{q:ifclass} using the \texttt{switch} construct instead of \texttt{if}.
		\item Consider why the solution to question \ref{sec:flowcontrol}.\ref{q:ifelse} does not lend itself to being rewritten using a \texttt{switch} statement. When would you use a \texttt{switch} statement instead of a series of \texttt{if}, \texttt{elseif}, \texttt{else} statements?
		\item
		\begin{enumerate}
			\item In a new script write a \texttt{for} loop that counts from 1 to 10. The script should output the value of the loop counter (to the screen) for each loop iteration.
			\item Add a nested loop within the first loop that counts from 10 to 1.
			\item Modify your code so both loops terminate when the loop counters are equal.
		\end{enumerate}		
		\item Write a while loop that repeatedly multiplies a variable \texttt{B} by 10 until \texttt{B} equals \texttt{inf}. If \texttt{B} is initially 1, how many loop iterations are required before the loop terminates?
	\end{enumerate}
	\subsection*{Requires some use of MATLAB's help}
	\begin{enumerate}[resume]
		\item Consider when you would use a while loop instead of a for loop (and vice-versa). Read the MATLAB help documentation if you're unsure (\texttt{doc for} and \texttt{doc while})
		\item Write a \texttt{for} loop which loops from 1 to 100 in steps of 1.
		\begin{enumerate}
			\item If the loop counter is divisible by both 5 and 7 (i.e. the answer is an integer), output suitable text to the screen. (hint: use the \texttt{mod} function and \texttt{fprintf})
			\item Count how many times the above condition is met and output the result
		\end{enumerate}
		\item Write a while loop which calculates the first 10 numbers which are divisible by 3, 4 and 5. From within your loop, output sensible text to the screen for each number which meets all these criteria.
		\item
		\begin{enumerate}
			\item  Write a script which requests the user to input a number, then uses a loop to test whether this is a prime number. You should output some text to indicate the result, and if the number is not prime, indicate by which number is it divisible. (hints: \texttt{fprintf}, \texttt{input})
			\item Make your code more robust by checking that the user input is
			\begin{enumerate}
				\item an integer
				\item greater than 1
			\end{enumerate}
			and halting execution with a helpful error message if the input is invalid (hint: \texttt{assert} that these conditions are true)
		\end{enumerate}
	\end{enumerate}
\end{document}
