\documentclass[a4paper]{article}

\usepackage{matlab-prettifier}
\usepackage[T1]{fontenc}
\usepackage{hyperref}
\hypersetup{
  colorlinks   = true, %Colours links instead of ugly boxes
  urlcolor     = blue, %Colour for external hyperlinks
  linkcolor    = black, %Colour of internal links
}
\usepackage{enumerate}

\title{MATLAB lesson 3: Matrices\\Exercise sheet}
\date{}
\author{Dr. Gerard Capes\thanks{Questions and feedback can be directed to \href{mailto:gerard.capes@manchester.ac.uk?subject=Feedback on MATLAB lesson 3 (matrices) exercise sheet}{gerard.capes@manchester.ac.uk}}}

\begin{document}
\maketitle

\section{Based on the lesson}
{\large These exercises are designed to test your understanding of the lesson content and can be completed by referring to the material in the lesson.}
\begin{enumerate}
	\item Create a matrix \textbf{a}, which is a 2*3 matrix (two rows, three columns) of ones
	\item Create a 4*3 matrix \textbf{b}, of uniformly distributed random numbers
	\item Create a 10*10 matrix \textbf{c}, of zeros
	\item Set row 3, column 4 of matrix \textbf{c} equal to 4
	\item Set row 2 equal to 5
	\item Set rows 5 to 7 between columns 6 and 8 equal to 6
	\item Set all elements of matrix c equal to 3
	\item Calculate the matrix-square of \textbf{c}, i.e. the matrix product of \textbf{c}*\textbf{c}
	\item Calculate the elementwise square of matrix c
	\item Divide every element in array \textbf{c} by 3
	\item Create a 4*4 magic array, \textbf{m}
	\item Find the minimum value of \textbf{m}
	\item Create the following matrix:
		\begin{table}[h!]
			\centering
			\begin{tabular}{ccc}
				1 & 2 & 3\\
				4 & 5 & 6\\
				7 & 8 & 9\\
			\end{tabular}
		\end{table}	
\end{enumerate}

\pagebreak

\section{Using the MATLAB documentation}
{\large This section will require you to search within the MATLAB help}

	\begin{enumerate}
	% Medium		
		\item  Find the indices of the minimum value of matrix \textbf{m} you previously created (consult the documentation for \texttt{min}).
	% Difficult	
		For example, if the minimum value is in row 2, column 3, their indices are (2,3).
		
		\item Create an array of random numbers with 500 rows and 1 column
		\item Test if \texttt{any} values are greater than
			\begin{enumerate}
				\item 0.5
				\item 0.9
				\item 0.99
			\end{enumerate}
		\item \texttt{Find} the indices where values are greater than 0.99
		\item Are \texttt{all} values are greater than
			\begin{enumerate}
				\item 0.5
				\item 0.1
				\item 0.01
			\end{enumerate}
	\end{enumerate}
\end{document}
